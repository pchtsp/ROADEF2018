\documentclass[a4paper,11pt]{article}
\usepackage[utf8]{inputenc}
\usepackage{amsmath}

%opening
\title{ROADEF2018: models}
\author{}

\setlength{\topmargin}{-2.1cm}
\setlength{\textwidth}{15.8cm}
\setlength{\textheight}{26.3cm}
\setlength{\oddsidemargin}{-0.1cm}
\setlength{\evensidemargin}{-0.1cm}

\def\II{{\mathcal{I}}}
\def\TT{{\mathcal{T}}}
\def\KK{{\mathcal{K}}}

\usepackage{graphicx}
\usepackage{amssymb}

\begin{document}

\maketitle

\section{Furini2016}

    \subsection{Hypothesis}

    \begin{itemize}
        \item All jumbos are the same.
        \item No deffects.
        \item No sequence between items.
    \end{itemize}

    \subsection{Notations}

    \textbf{Sets}

    \begin{tabular}{ll}
        $l \in \mathcal{L}$ & Set of possible levels. There are the following: 0, 1, 2, 3. \\
        $j \in \mathcal{J}$ & Set of possible plates. A plate is defined as a tuple with a width $w$ and a length $l$. \\
        $j'_j \in \mathcal{J}$ & inverse plate of $j$. So width and length are exchanged. \\
        $C^o_{qjl}$ & set of tuples ($oqjl$). All combinations of possible cuts at distance $q$ that can be done in plate $j$ at level $l$ and orientation $o$. \\
        % $J^o_{qj0} \subset C^o_{qjm}$ & set of of possible cuts on plate at level 0 ($m=0$).
        $J_{l}$ & all plates that can be produced at level $l$.\\
        $L_{j}$ & all levels where plate $j$ can be produced.\\
        % $J_1$ & all plates that can be produced at level 1 ($J_1$).
        $C'_{km} \subset C^o_{qjl}$ & for each plate $k$ in level $m$, the set of available cuts that produce this piece. This is calculated from parameter $a^o_{qkljm}$, see below. \\
        $C''_{jl} \subset C^o_{qjl}$ & for each plate $j$ in level $l$, the set of available cuts that are used by this cut. This is calculated from set $C^o_{qjl}$. \\
    \end{tabular}

    \vskip 0.3cm

    \subsection{Parameters}

    \begin{tabular}{ll}
        $a^o_{qkmjl}$ & plate $k$, at level $m$ is produced while cutting plate $j$ at level $l$ in position $q$ with orientation $o$. \\
        $W_j$ & width of plate $j$.\\
        $d_j$ & height of plate $j$.\\
    \end{tabular}

    \subsection{Variables}

    \begin{itemize}
         \item $x^o_{lqj}$: number of times cut of distance $q$ is done in a plate of type $j$ at level $l$ with orientation $o$.
         \item $y_{jl}$: number of times a plate of type $j$ has been used to cover demand.
         \item $r_{jl}$: number of times a plate of type $j$ has been used as residue (right side of each plate). Only exists for level 1 ($l=1$).
    \end{itemize}

    \subsection{Model}

    \begin{align}
        & \text{Min}\; \sum_{oqjl \in C^o_{qj0}} x^o_{lqj} W_j - \sum_{j \in J_1} r_{j1} W_j
    \end{align}

    \begin{align}
        & \sum_{oqkm \in C'_{jl}} x^o_{oqkm} \geq \sum_{oqjl \in C''_{jl}} x^o_{oqjl} + y_{jl} + r_{jl}
            & l \in \mathcal{L}, j \in J_l \\
        & \sum_{l \in L_{j}} y_{jl} + y_{j'l} \geq d_j & j \in \mathcal{J} \\
        & x_{i} \geq 0,\; integer \\
    \end{align}

\end{document}

